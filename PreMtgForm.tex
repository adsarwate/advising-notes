\documentclass[10pt, letter]{article}

% margins and fonts
\usepackage[letterpaper, margin=1in]{geometry}
\usepackage{libertinus}
\usepackage[T1]{fontenc}
%\usepackage[noTeX]{mmap} % better ligatures and unicode


% math fonts etc
\usepackage{amsmath,amssymb,amsthm}
\usepackage{mathtools} % for xlongleftrightarrow
\usepackage{mleftright}  % fixes some annoying spacing issues
\usepackage{bm} % boldface math

% graphics and figures
\usepackage{xcolor,graphicx}
%\usepackage{tikz,tkz-euclide}
%\usetikzlibrary{backgrounds, shapes.symbols, patterns, arrows}
\usepackage{pgfplots}
\pgfplotsset{compat=1.17}
\usepackage{float,subfigure}

% customized environments
\usepackage[shortlabels]{enumitem}
\usepackage{mdframed}
\usepackage[breakable]{tcolorbox}




% Bibliography stuff, citations, etc.
\usepackage[bibstyle=numeric,backend=biber]{biblatex}
\definecolor{OI-vermillion}{RGB}{213,94,0}
\usepackage[colorlinks=true,citecolor=OI-vermillion,urlcolor=OI-vermillion]{hyperref}
\usepackage[nameinlink]{cleveref}

% theorems etc (also  problems)
\usepackage{thmtools}
\usepackage{thm-restate}
\declaretheorem[name=Definition,numberwithin=section]{definition}
\declaretheorem[name=Theorem,numberwithin=section]{theorem}
\declaretheorem[name=Lemma,sibling=theorem]{lemma}
\declaretheorem[name=Proposition,sibling=theorem]{proposition}
\declaretheorem[style=remark]{remark}
\declaretheorem[name=Claim]{claim}
\declaretheorem[name=Conjecture]{conjecture}
\declaretheorem[name=Problem]{problem}




% Standard handout
\newcommand{\handout}[5]{%
{\parindent 0pt
#1 \hfill #2 \\
#3 \hfill #4} \\
\begin{center}
\hrule\vspace*{10pt}
{#5}
\vspace*{10pt}\hrule
\end{center}
\vspace*{.5cm}}

\newcommand{\sect}[1]{
\vspace*{12pt}
\noindent \mbox{\large #1}}




% header environment
\newmdenv[backgroundcolor=white,leftline=false,rightline=false,leftline=false]{header}

% example environment
\newmdenv[backgroundcolor=blue!3,frametitle={Example:}]{example}

% warning environment
\newmdenv[backgroundcolor=red!3,frametitle={Note:}]{note}

% solution environment
\newmdenv[
        linewidth=1pt,%
        backgroundcolor=black!3
        ]
    {sol}
\newenvironment{solution}[1][]
    {\begin{sol}[frametitle={Solution authored by #1}]}
    {\end{sol}}


% rubric environment
\newmdenv[backgroundcolor=yellow!30,frametitle={Rubric:}]{rubric}

% toggles/cases
%\newtoggle{solution}
%\newtoggle{booklet}

\mathsurround 1.5pt
\def\s{\sigma}
\def\lbar{\overline\ell}
\def\mc{\mathcal}
\def\mbf{\mathbf}
\def\mbs{\boldsymbol}
\def\msf{\mathsf}

\pgfmathdeclarefunction{gauss}{2}{%
  \pgfmathparse{1/(#2*sqrt(2*pi))*exp(-((x-#1)^2)/(2*#2^2))}%
}


%% editing macros
\newcommand{\ads}[1]{\textcolor{blue}{\textbf{ADS:} {\em #1}}}
\usepackage[normalem]{ulem}
\newcommand\redout{\bgroup\markoverwith{\textcolor{red}{\rule[.5ex]{2pt}{0.4pt}}}\ULon}
\newcommand{\edit}[2]{\redout{#1}{~\textcolor{blue}{#2}}}


%% Language macros
\newcommand{\Ie}{\textit{i.e., }}
\newcommand{\Eg}{\textit{e.g., }}

\newcommand{\defeq}{\triangleq} % req. amssymb package



%%% delimiters: \foo{} is non-autosizing but \foo*{} is autosizing
\DeclarePairedDelimiter\parens{\lparen}{\rparen}  
\DeclarePairedDelimiter\abs{\lvert}{\rvert}
\DeclarePairedDelimiter\norm{\lVert}{\rVert}
\DeclarePairedDelimiter\floor{\lfloor}{\rfloor}
\DeclarePairedDelimiter\ceil{\lceil}{\rceil}
\DeclarePairedDelimiter\braces{\lbrace}{\rbrace}
\DeclarePairedDelimiter\bracks{\lbrack}{\rbrack}
\DeclarePairedDelimiter\angles{\langle}{\rangle}
\newcommand{\set}[1]{\bracks*{#1}}

%%% Special norms and linear algebra stuff
\newcommand{\subgnorm}[1]{\norm*{#1}_{\psi_2}}
\newcommand{\subexpnorm}[1]{\norm*{#1}_{\psi_1}}
\newcommand{\frobnorm}[1]{\norm*{#1}_{\mathrm{F}}}
\newcommand{\opnorm}[1]{\norm*{#1}_{\mathrm{op}}}
\newcommand{\Lipnorm}[1]{\norm*{#1}_{\mathrm{Lip}}}
\DeclareMathOperator{\eig}{\mathrm{eig}}


%%% Triple bar delimiter for matrix norms
\DeclareFontFamily{U}{matha}{\hyphenchar\font45}
\DeclareFontShape{U}{matha}{m}{n}{
      <5> <6> <7> <8> <9> <10> gen * matha
      <10.95> matha10 <12> <14.4> <17.28> <20.74> <24.88> matha12
      }{}
\DeclareSymbolFont{matha}{U}{matha}{m}{n}
\DeclareFontSubstitution{U}{matha}{m}{n}

% Math symbol font mathb
\DeclareFontFamily{U}{mathx}{\hyphenchar\font45}
\DeclareFontShape{U}{mathx}{m}{n}{
      <5> <6> <7> <8> <9> <10>
      <10.95> <12> <14.4> <17.28> <20.74> <24.88>
      mathx10
      }{}
\DeclareSymbolFont{mathx}{U}{mathx}{m}{n}
\DeclareFontSubstitution{U}{mathx}{m}{n}

% Symbol definition to get matrix norms
\DeclareMathDelimiter{\vvvert}{0}{matha}{"7E}{mathx}{"17}
\DeclarePairedDelimiterX{\matnorm}[1]
  {\vvvert}
  {\vvvert}
  {\ifblank{#1}{\:\cdot\:}{#1}}






%%% Special functions that use delimiters
\newcommand{\ip}[2]{\left\langle #1, #2 \right\rangle}
\newcommand{\kldiv}[2]{D\parens*{#1 \| #2}}
\newcommand{\trn}{^\intercal} % operator transpose
\newcommand{\inv}{^{-1}} %inverse
\DeclareMathOperator{\argmax}{\mathrm{argmax}}
\DeclareMathOperator{\argmin}{\mathrm{argmin}}
\DeclareMathOperator{\diag}{\mathrm{diag}}
\DeclareMathOperator{\tr}{\mathop{\mathrm{tr}}\nolimits}
\DeclareMathOperator{\rank}{\mathrm{rank}\nolimits}

% number systems
\newcommand{\R}{\mathbb{R}}
\newcommand{\C}{\mathbb{C}}
\newcommand{\N}{\mathbb{N}}
\newcommand{\Z}{\mathbb{Z}}
\newcommand{\F}{\mathbb{F}}
\newcommand{\Q}{\mathbb{Q}}
% deprecate these
\newcommand{\bbR}{\mathbb{R}}
\newcommand{\bbC}{\mathbb{C}}
\newcommand{\bbF}{\mathbb{F}}

%%% GEOMETRY
\DeclareMathOperator{\Vol}{\mathrm{Vol}}   % volume
\DeclareMathOperator{\Surf}{\mathrm{Surf}} % surface area

%%% STATISTICS AND PROBABILITY
\newcommand{\Var}{\mathop{\mathrm{Var}}\nolimits}
\newcommand{\Cov}{\mathop{\mathrm{Cov}}\nolimits}
\newcommand{\MSE}{\mathop{\mathrm{MSE}}\nolimits}
\newcommand{\LLR}{\mathop{\mathrm{LLR}}\nolimits}
\newcommand{\FWER}{\mathop{\mathrm{FWER}}\nolimits}

\newcommand{\cX}{{\cal X}}
\newcommand{\cY}{{\cal Y}}
\newcommand{\E}[1]{\mathbb{E}\bracks*{#1}}
\newcommand{\condE}[2]{\mathbb{E}\bracks*{#1 \mid #2}}
\renewcommand{\P}[1]{\mathbb{P}\parens*{#1}}
\newcommand{\condP}[2]{\mathbb{P}\parens*{#1 \mid #2}}

\newcommand{\Bern}{\mathsf{Bern}}
\newcommand{\Unif}{\mathsf{Unif}}
\newcommand{\Expv}{\mathsf{Exp}}
\newcommand{\Lap}{\mathsf{Lap}}
\newcommand{\Poi}{\mathsf{Poi}}
\newcommand{\Gamv}{\mathsf{Gamma}}
\newcommand{\Dirv}{\mathsf{Dir}}
\newcommand{\Mult}{\mathsf{Mult}}
\newcommand{\Beta}{\mathsf{Beta}}
\newcommand{\Geomv}{\mathsf{Geom}}
\newcommand{\Binomv}{\mathsf{Binom}}
\newcommand{\NegBinomv}{\mathsf{NB}}
\newcommand{\Weibull}{\mathsf{Weibull}}

\newcommand{\iidsim}{\stackrel{\mathrm{i.i.d.}}{\sim}}
\newcommand{\bht}{\underset{\mc{H}_0}{\overset{\mc{H}_1}{\gtrless}}}
\newcommand{\thb}{\underset{\mc{H}_1}{\overset{\mc{H}_0}{\gtrless}}}
\newcommand{\bhtk}{\underset{\mc{H}_{0,k}}{\overset{\mc{H}_{1,k}}{\gtrless}}}
\newcommand{\bhtNeg}{\underset{\mc{H}_-1}{\overset{\mc{H}_1}{\gtrless}}}
%\newcommand{\bhtK}{\underset{\mc{H}_{0,k}}{\overset{\mc{H}_{1,k}}{\gtrless}}}

\newcommand{\Pfa}{P_{\mathrm{FA}}}
\newcommand{\Pmd}{P_{\mathrm{MD}}}
\newcommand{\Pd}{P_{\mathrm{D}}}
\newcommand{\Pfak}{P_{\mathrm{FA},k}}
\newcommand{\Pmdk}{P_{\mathrm{MD},k}}

\newcommand{\dbayes}{\delta_{\mathrm{B}}}
\newcommand{\dmap}{\delta_{\mathrm{MAP}}}
\newcommand{\dml}{\delta_{\mathrm{ML}}}
\newcommand{\dminimax}{\delta_{\mathrm{m}}}
\newcommand{\dnp}{\delta_{\mathrm{NP}}}
\newcommand{\Dnp}{\Delta_{\mathrm{NP}}}
\newcommand{\Rmax}{R_{\mathrm{max}}}


% plotting macros
%\pgfmathdeclarefunction{mygauss}{2}{%
%  \pgfmathparse{1/(#2*sqrt(2*pi))*exp(-((x-#1)^2)/(2*#2^2))}%
%}


\begin{document}


\handout{Rutgers ECE}{2022}{\textcolor{red}{your name here}}{\textcolor{red}{date of meeting}}{Pre-Meeting Check-In and Agenda}

\noindent \textit{Try to fill this form out as you work during the week. Examples are given in the template so have a sense of what kinds of things to write. The goal is to become more self-aware about your own work: what were you trying to do, what did you do, what worked, what didn't work, and why. Sometimes you don't get anything done in a week because of life intervening! Don't worry about it. Try to send it the day \textbf{before} your meeting so there is time for your advisor to read it and figure out how best to help.}

\subsubsection*{What were your main goals since the last meeting?}

\textit{You should have them from the post-meeting form, but they also may have changed.}


\begin{note}
	\begin{enumerate}
	\item Watch video lectures to get a basic understanding of RMT.
	\item Download and look at references on Matrix Bernstein inequalities used for compressed sensing 
	\item See if existing results apply to my problem and if not, why not.
	\item Search for a better parameters in the simulation.
	\item Download the paper template for the conference and check due dates.
	\end{enumerate}
\end{note}


\subsubsection*{What did you do to achieve these goals?}

\textit{You can log these as you do them instead of trying to remember 6 days later.}

\begin{note}
I didn't feel well on Wednesday.
	\begin{enumerate}
	\item Monday,Tuesday - watched first 4 lectures on YouTube, started skimming notes.
	\item Tuesday - Downloaded some references. Th - looked through two older papers (one by Tropp and one by Vershynin)
	\item Thursday - Did not see how those results are at all related but will try today.
	\item Monday - Got some parameters that seem to work locally, but the cluster is down for maintenance so didn't get to run the simulations. Maybe it is up today?
	\item Tuesday- Downloaded the template into an overleaf: \url{https://www.overleaf.com/blahblah}
	\end{enumerate}
\end{note}


\subsubsection*{Did you find something interesting or new?}  

\textit{A good day is a day where you learn something new!}

\begin{note}
	\begin{itemize}
	\item I didn't know there were so many variations on the Matrix Bernstein Inequality!
	\item It wasn't on my list, but I saw this paper online and it looked interesting: \url{https://www....}
	\item I also learned about the Wishart distribution, which might be useful?
	\end{itemize}
\end{note}


\subsubsection*{If you got stuck, where did you get stuck? What did you try to do in order get unstuck?}

\textit{A lot of research is hitting barriers or dead-ends. Reflecting on what you tried and writing it down can help you identify why it didn't work. Knowing what you tried helps to avoid repeating things and also lets your advisor suggest better options. }

\begin{note}
	\begin{itemize}
	\item I'm not sure how to prioritize reading all these matrix concentration papers. I tried going oldest to newest but it's hard because the notation is not consistent.
	\item The assumptions for the matrix concentration results are confusing.  In particular in Lemma 4 of Vershynin's paper, he assumes this ``non-degeneracy" thing (Assumption 2) which I don't have a good intuition for. I looked in other papers (Xu et al. (2003) and Baron et al. (2005)) and they have different assumptions.
	\item In this other paper we had read before (Lee et al.) there are some bounds which I am not sure are correct, such as equations (16) and (18) on page 7.  The main theorem (Theorem 1) doesn't make sense to me -- what is $\beta$? I could not find it explained anywhere in the paper, either in the published or ArXiV versions.
	\item The cluster is down for maintenance. I looked on the website but there was no announcement there. Is there a mailing list where announcements are made?
	\item I think I need to use a different data set for experiments but I'm not sure which one is the best. There are 5 data sets used in Lee et al. so should we use the same ones?
	\end{itemize}
\end{note}


\subsubsection*{What do you want to talk about at the meeting?}

\textit{Going in with an agenda makes sure you get to discuss all the things you wanted to figure out. }

\begin{note}
	\begin{enumerate}
	\item Questions on how to map existing concentration inequalities to our problem.
	\item Make a plan regarding datasets/experiments. 
	\item Discuss course selection for next semester.
	\end{enumerate}
\end{note}


\end{document}
