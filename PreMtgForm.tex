\documentclass[10pt, letter]{article}

% margins and fonts
\usepackage[letterpaper, margin=1in]{geometry}
\usepackage{libertinus}
\usepackage[T1]{fontenc}
%\usepackage[noTeX]{mmap} % better ligatures and unicode


% math fonts etc
\usepackage{amsmath,amssymb,amsthm}
\usepackage{mathtools} % for xlongleftrightarrow
\usepackage{mleftright}  % fixes some annoying spacing issues
\usepackage{bm} % boldface math

% graphics and figures
\usepackage{xcolor,graphicx}
%\usepackage{tikz,tkz-euclide}
%\usetikzlibrary{backgrounds, shapes.symbols, patterns, arrows}
\usepackage{pgfplots}
\pgfplotsset{compat=1.17}
\usepackage{float,subfigure}

% customized environments
\usepackage[shortlabels]{enumitem}
\usepackage{mdframed}
\usepackage[breakable]{tcolorbox}




% Bibliography stuff, citations, etc.
\usepackage[bibstyle=numeric,backend=biber]{biblatex}
\definecolor{OI-vermillion}{RGB}{213,94,0}
\usepackage[colorlinks=true,citecolor=OI-vermillion,urlcolor=OI-vermillion]{hyperref}
\usepackage[nameinlink]{cleveref}

% theorems etc (also  problems)
\usepackage{thmtools}
\usepackage{thm-restate}
\declaretheorem[name=Definition,numberwithin=section]{definition}
\declaretheorem[name=Theorem,numberwithin=section]{theorem}
\declaretheorem[name=Lemma,sibling=theorem]{lemma}
\declaretheorem[name=Proposition,sibling=theorem]{proposition}
\declaretheorem[style=remark]{remark}
\declaretheorem[name=Claim]{claim}
\declaretheorem[name=Conjecture]{conjecture}
\declaretheorem[name=Problem]{problem}




% Standard handout
\newcommand{\handout}[5]{%
{\parindent 0pt
#1 \hfill #2 \\
#3 \hfill #4} \\
\begin{center}
\hrule\vspace*{10pt}
{#5}
\vspace*{10pt}\hrule
\end{center}
\vspace*{.5cm}}

\newcommand{\sect}[1]{
\vspace*{12pt}
\noindent \mbox{\large #1}}




% header environment
\newmdenv[backgroundcolor=white,leftline=false,rightline=false,leftline=false]{header}

% example environment
\newmdenv[backgroundcolor=blue!3,frametitle={Example:}]{example}

% warning environment
\newmdenv[backgroundcolor=red!3,frametitle={Note:}]{note}

% solution environment
\newmdenv[
        linewidth=1pt,%
        backgroundcolor=black!3
        ]
    {sol}
\newenvironment{solution}[1][]
    {\begin{sol}[frametitle={Solution authored by #1}]}
    {\end{sol}}


% rubric environment
\newmdenv[backgroundcolor=yellow!30,frametitle={Rubric:}]{rubric}

% toggles/cases
%\newtoggle{solution}
%\newtoggle{booklet}

\usepackage{\bm}

% shortcuts for 
\def\mc{\mathcal}
\def\mbb{\mathbb}
\def\msf{\mathsf}
%\def\mbf{\mathbf} -- deprecated: use \bm
%\def\mbs{\boldsymbol}

% inner products and norms
\newcommand{\ip}[2]{\left\langle #1,\ #2 \right\rangle}
\newcommand{\norm}[1]{\left\| #1 \right\|}

% common functionals
\newcommand{\det}{\mathop{\mathrm{det}}\nolimits}
\newcommand{\tr}{\mathop{\mathrm{Tr}}\nolimits}
\newcommand{\sgn}{\mathop{\mathrm{sgn}}\nolimits}
\newcommand{\diag}{\mathop{\mathrm{diag}}\nolimits}
\newcommand{\Var}{\mathop{\mathrm{Var}}\nolimits}
\newcommand{\rect}{\mathop{\mathrm{rect}}\nolimits}
\newcommand{\sinc}{\mathop{\mathrm{sinc}}\nolimits}

% optimization stuff
\newcommand{\argmax}{\mathop{\mathrm{argmax}}}
\newcommand{\argmin}{\mathop{\mathrm{argmin}}}
\newcommand{\argsup}{\mathop{\mathrm{argsup}}}
\newcommand{\arginf}{\mathop{\mathrm{arginf}}}

% probability stuff
\newcommand{\E}{\mathbb{E}}
\renewcommand{\P}{\mathbb{P}}

\newcommand{\iidsim}{\stackrel{\mathrm{i.i.d.}}{\sim}}

\newcommand{\bern}{\mathsf{Bernoulli}}
\newcommand{\unif}{\mathsf{Uniform}}
\newcommand{\expv}{\mathsf{Exp}}
\newcommand{\binomv}{\mathsf{Binomial}}
\newcommand{\lap}{\mathsf{Laplace}}

% information theory stuff
\newcommand{\bsc}{\mathsf{BSC}}
\newcommand{\bec}{\mathsf{BEC}}
\newcommand{\mi}[2]{I\left( #1 ; #2 \right)}
\newcommand{\kldiv}[2]{D\left( #1 \middle| #2 \right)}

% Editing macros -- add your own!
\newcommand{\todo}[1]{\textcolor{red} {\textbf{Todo: #1}}}
\newcommand{\ads}[1]{\textcolor{BurntOrange}{\textbf{ADS:} #1}}

\usepackage[normalem]{ulem}
\newcommand\redout{\bgroup\markoverwith{\textcolor{red}{\rule[.5ex]{2pt}{0.4pt}}}\ULon}
\newcommand{\edit}[2]{\redout{#1}{~\textcolor{blue}{#2}}}

\newcommand{\cn}{\textcolor{red}{[\raisebox{-0.2ex}{\tiny\shortstack{citation\\[-1ex]needed}}]}}




\begin{document}


\handout{Rutgers ECE Graduate Research}{2022}{your name here}{Date: \today}{Pre-Meeting Check-In and Agenda}


\noindent \textit{The purpose of these questions is to help you get the most out of a meeting. By writing down the answers to these questions you should get a better sense of what you were trying to accomplish, how much time you spent trying to accomplish those things, what you tried, what worked, and what didn't work.  That also helps me figure out how you're stuck -- sometimes it's a matter of learning some facts or reference, other times it is trying a different strategy for research.  Sometimes you don't get anything done in a week because of life intervening!  Don't worry about it. If you are honest with yourself and communicate openly, it will be easier to find the best way to help. Examples are given in the template so have a sense of what kinds of things to write.}


\subsubsection*{What were your main goals?}

\begin{note}
	\begin{enumerate}
	\item Watch video lectures to get a basic understanding of RMT.
	\item Download and look at references on Matrix Bernstein inequalities used for compressed sensing 
	\item See if existing results apply to my problem and if not, why not.
	\item Search for a better parameters in the simulation.
	\item Download the paper template for the conference and check due dates.
	\end{enumerate}
\end{note}


\subsubsection*{What did you do to achieve these goals?}

\begin{note}
I didn't feel well on Wednesday.
	\begin{enumerate}
	\item M,T - watched first 4 lectures on YouTube, started skimming notes.
	\item T - Downloaded some references. Th - looked through two older papers (one by Tropp and one by Vershynin)
	\item Th - Did not see how those results are at all related but will try today.
	\item M - Got some parameters that seem to work locally, but the cluster is down for maintenance so didn't get to run the simulations. Maybe it is up today?
	\item T- Downloaded the template into an overleaf: \url{https://www.overleaf.com/blahblah}
	\end{enumerate}
\end{note}


\subsubsection*{Did you find something interesting or new?}  

\begin{note}
	\begin{itemize}
	\item I didn't know there were so many variations on the Matrix Bernstein Inequality.
	\item It wasn't on my list, but I saw this paper online and it looked interesting: \url{https://www....}
	\item I also learned about the Wishart distribution, which might be useful?
	\end{itemize}
\end{note}


\subsubsection*{If you got stuck, where did you get stuck? What did you try?}

\begin{note}
	\begin{itemize}
	\item I'm not sure how to prioritize reading all these matrix concentration papers. I tried going oldest to newest but it's hard because the notation is not consistent.
	\item The assumptions for the matrix concentration results are confusing.  In particular in Lemma 4 of Vershynin's paper, he assumes this ``non-degeneracy" thing (Assumption 2) which I don't have a good intuition for.
	\item In this other paper we had read before (Lee et al.) there are some bounds which I am not sure are correct, such as equations (16) and (18) on page 7.  The main theorem (Theorem 1) doesn't make sense to me -- what is $\beta$? I could not find it explained anywhere in the paper.
	\item The cluster is down. I looked on the website but there was no announcement there. Is there a mailing list?
	\item I think I need to use a different data set for experiments but I'm not sure which one is the best. There are 5 data sets used in Lee et al. so should we use the same ones?
	\end{itemize}
\end{note}


\subsubsection*{Please give an agenda for the meeting}

\begin{note}
	\begin{enumerate}
	\item Questions on how to map existing concentration inequalities to our problem.
	\item Make a plan regarding datasets/experiments. 
	\item Discuss course selection for next semester.
	\end{enumerate}
\end{note}


\end{document}
